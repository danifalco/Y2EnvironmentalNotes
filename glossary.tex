% ========================= GLOSSARY =========================
\newglossaryentry{irradiance}{
  name={Irradiance},
  description={The amount of power per unit area incident on a surface. It is measured in $Wm^{-2}$ and for this course
  it will always be power from EM radiation}
}

\newglossaryentry{albedo}{
  name={Planetary Albedo},
  description={The fraction of solar radiation that is reflected back into space. It is a dimensionless quantity. It 
  goes from 0 to 1. See section \ref{sec:albedo} for more details}
}

\newglossaryentry{TSI}{
  name={TSI},
  description={($\alpha_p$) Total Solar Irradiance. It is the amount of solar irradiance emitted by the Sun. The sum 
  total of photons emitted by it, converted to $Wm^{-2}$}
}

\newglossaryentry{TOA}{
  name={TOA},
  description={Top of Atmosphere. This is the top of the atmosphere, where the atmosphere ends and space begins}
}

\newglossaryentry{LW}{
  name={LW},
  description={Long wave. We will use this to refer to long wave EM radiation}
}

\newglossaryentry{SW}{
  name={SW},
  description={Short wave. We will use this to refer to short wave EM radiation}
}

\newglossaryentry{emissivity}{
  name={Emissivity},
  description={The ratio of the energy radiated by a surface to the energy radiated by a black body at the same 
  temperature. It is a dimensionless quantity. It goes from 0 to 1}
}

\newglossaryentry{absorptivity}{
  name={Absorptivity},
  description={The ratio of the energy absorbed by a surface to the energy incident on it. It is a dimensionless 
  quantity. It goes from 0 to 1}
}

\newglossaryentry{transmissivity}{
  name={Transmissivity},
  description={The ratio of the energy transmitted by a surface to the energy incident on it. It is a dimensionless 
  quantity. It goes from 0 to 1}
}

\newglossaryentry{OLR}{
  name={OLR},
  description={Outgoing Long Wave Radiation. The amount of LW radiation emitted by the Earth. It is measured in 
  $Wm^{-2}$}
}

\newglossaryentry{forcing}{
  name={Radiative Forcing},
  description={The change in net irradiance at the TOA due to a change in some external factor. It is measured in 
  $Wm^{-2}$}
}


\newglossaryentry{opticaldepth}{
    name={Optical Depth},
    description={The measure of how transparent a medium is to electromagnetic radiation. It is a dimensionless quantity.
    See section \ref{sec:opticaldepth} for more details}
}

\newglossaryentry{feedbackparam}{
    name={Climate Feedback Parameter},
    description={A mesure of how much the climate system responds to a change in some surface temperature change. A 
    measure of how much energy is trapped or lost as a result of this change in surface temperature. See \ref{eq:gamma}}
}

\newglossaryentry{lapse_rate}{
    name={Lapse Rate},
    description={The negative rate of change of temperature with height. It is measured in $Km^{-1}$. See section
    \ref{sec:lapse_rate} for more details}
}

% Gross Primary Productivity (GPP)
\newglossaryentry{gpp}{
    name={Gross Primary Productivity (GPP)},
    description={The total amount of carbon fixed by photosynthesis in an ecosystem per unit time}
}

% Net Primary Productivity (NPP)
\newglossaryentry{npp}{
    name={Net Primary Productivity (NPP)},
    description={The GPP minus the amount of carbon released by respiration i.e. $NPP = GPP - R$}
}

% Dissolved Inorganic Carbon (DIC)
\newglossaryentry{dic}{
    name={Dissolved Inorganic Carbon (DIC)},
    description={The sum of dissolved carbon dioxide, carbonic acid, bicarbonate and carbonate}
}

% Photosynthetically Active Radiation (PAR)
\newglossaryentry{par}{
    name={Photosynthetically Active Radiation (PAR)},
    description={The spectral range of solar radiation from 400 to 700 nm that photosynthetic organisms are able to use
    in the process of photosynthesis. See figure \ref{fig:PAR_spectrum}}
}

% Ocean Biological Carbon Pump
\newglossaryentry{obcp}{
    name={Ocean Biological Carbon Pump},
    description={The process by which carbon is transported from the surface ocean to the deep ocean. See section 
    \ref{sec:ocean-biological-carbon-pump}}
}

\newglossaryentry{well_mixed_greenhouse_gas}{
    name={Well-Mixed Greenhouse Gas},
    description={A gas that is well-mixed in the atmosphere (troposphere), i.e. its concentration does not vary
    significantly with geographical location or height. See section \ref{sec:well-mixed-greenhouse-gases} for more
    details}
}

\newglossaryentry{radiative_efficiency}{
    name={Radiative Efficiency},
    description={The ability of a forcing agent to cause a change in radiative forcing per unit change in concentration
    or amount thereof}
}

\newglossaryentry{global_warming_potential}{
    name={Global Warming Potential},
    description={A metric that provides a means to compare the effectiveness of different agents to that of \ce{CO2}.
    See section \ref{sec:comp-impact-diff} for more details}
}

\newglossaryentry{aerosol}{
    name={Aerosol},
    description={Small liquid drops or particulates suspended in the atmosphere. They can be natural or anthropogenic in
    origin and can be made up of many different constituents. They can be emitted directly into the atmosphere (primary)
    or formed in the atmosphere as a result of chemical processes (secondary). See section \ref{sec:aerosol-radiative-forcing}
    for more details}
}

% Thermal inertia
\newglossaryentry{thermal_inertia}{
    name={Thermal Inertia},
    description={The property of a material to resist changes in temperature. 
    The higher the thermal inertia, the longer it takes for the material to 
    reach a new equillibrium temperature}
}

% Net Feedback Parameter
\newglossaryentry{net_feedback_parameter}{
    name={Net Feedback Parameter},
    description={The sum of all feedback parameters due to all feedback processes
    operating in the climate system. See section \ref{sec:climate-sensitivity} 
    for more details}
}

% Adjustment Timescale
\newglossaryentry{adjustment_timescale}{
    name={Adjustment Timescale},
    description={The time it takes for the climate system to reach a new 
    equillibrium temperature after a change in forcing. See section 
    \ref{sec:adjustment-timescale} for more details}
}

% Climate Sensitivity
\newglossaryentry{climate_sensitivity}{
    name={Climate Sensitivity},
    description={The equillibrium global mean surface temperature change in 
    response to a doubling of \ce{CO2} concentration. See section 
    \ref{sec:climate-sensitivity} for more details}
}

% Runaway Greenhouse Effect
\newglossaryentry{runaway_greenhouse_effect}{
    name={Runaway Greenhouse Effect},
    description={A theoretical scenario in which the climate system is unable to
    reach an equillibrium as a response to a forcing, i.e. the planet is 
    unable to ever radiate enough 
    energy to space to counteract the effect of the original forcing and all
    the operating feedbacks. See the end of section \ref{sec:climate-sensitivity}
    for more details}
}

% Section 8 glossary


\newglossaryentry{accuracy}{
    name={Accuracy},
    description={The degree to which the result of a measurement, calculation, or
    specification conforms to the correct value or a standard}
}

\newglossaryentry{precision}{
    name={Precision},
    description={The degree to which repeated measurements under unchanged 
    conditions show the same results}
}

% SST
\newglossaryentry{SST}{
    name=SST,
    description={Sea Surface Temperature}
}

% UHT
\newglossaryentry{UHI}{
    name=UHI,
    description={Urban Heat Island Effect. See section \ref{sec:uhi} for details}
}

% radiosonde
\newglossaryentry{radiosonde}{
    name=Radiosonde,
    description={An instrument that is attached to weather balloons and
    measure atmospheric parameters (temperature and humidity) as they ascend through
    the atmosphere}
}

% Beer-Lambert Law
\newglossaryentry{beer_lambert}{
    name={Beer-Lambert Law},
    description={The law that describes the attenuation of light through a medium.
    See section \ref{sec:lw_atm} for more details}
}

% Kirchoff's Law
\newglossaryentry{kirchoff}{
    name={Kirchoff's Law},
    description={The law that states that the emissivity of a body is equal to its
    absorptivity at a given wavelength and temperature so long as it is in local
    thermodynamic equilibrium}
}

% LTE
\newglossaryentry{LTE}{
    name={LTE},
    description={Local Thermodynamic Equillibrium - the state of a system in
    which all parts of the system are at the
    same temperature and there are no net flows of energy between different parts
    of the system}
}

% Schwarzschild Equation
\newglossaryentry{schwarzschild}{
    name={Schwarzschild Equation},
    description={The equation that describes the radiance seen by a satellite
    as a function of the radiance emitted by the surface and the radiance emitted
    by the atmosphere. See section \ref{sec:lw_atm} for more details}
}

% Weighting Function
\newglossaryentry{weigthing_func}{
    name={Weighting Function},
    description={The function that describes the contribution of a particular
    layer of the atmosphere to the radiance seen by a satellite. See section
    \ref{sec:weighting_functions} for more details}
}

% Section 9 glossary

% Energy balance model
\newglossaryentry{EBM}{
    name={EBM},
    description={Energy Balance Model. A model that describes the temperature
    of a planet as a function of the energy it receives from the Sun and the
    energy it radiates back into space. See section \ref{sec:EBM} for more details}
}

% IPCC
\newglossaryentry{IPCC}{
    name={IPCC},
    description={Intergovernmental Panel on Climate Change. A scientific and
    intergovernmental body under the auspices of the United Nations, set up at
    the request of member governments, dedicated to the task of providing the
    world with an objective, scientific view of climate change and its political
    and economic impacts. It does not produce its own research, rather it gathers
    insight from the scientific community and summarises it in its Assessment
    Reports}
}

% Section 10 Glossary

% SDG
\newglossaryentry{SDG}{
    name={Sustainable Development Goals},
    description={A list of 17 goals set by the United Nations in 2015 to be
    achieved by 2030. They are a call for action by all countries to promote
    prosperity while protecting the planet}
}

% SSP
\newglossaryentry{SSP}{
    name={Shared Socioeconomic Pathways},
    description={A set of five narratives that describe alternative future
    developments in the global society, demographics, economy, energy and
    land use systems. They are used as input for climate models to project
    future climate change and its impacts. See section \ref{sec:ssps_rcps} for 
    more details}
}

% RCP
\newglossaryentry{RCP}{
    name={Representative Concentration Pathways},
    description={A set of four greenhouse gas concentration trajectories
    adopted by the IPCC for its fifth Assessment Report. They are used as
    input for climate models to project future climate change and its impacts.
    See section \ref{sec:ssps_rcps} for more details}
}

% INDC
\newglossaryentry{INDC}{
    name={Intended Nationally Determined Contributions},
    description={The climate actions that countries intend to take under the
    Paris Agreement. They are not legally binding but it is a legal obligation
    for the countries to submit an NDC every 5 years}
}

% Kaya identity
\newglossaryentry{kaya}{
    name={Kaya Identity},
    description={An equation that expresses the total emissions of \ce{CO2}
    as the product of four factors: population, GDP per capita, energy intensity
    and carbon intensity. See section \ref{sec:achieve_climate_goals} for more details}
}

% Beltz Limit
\newglossaryentry{beltz_limit}{
    name={Beltz Limit},
    description={The maximum possible efficiency of a wind turbine, $\sim$59\%}
}

% Performance Coefficient
\newglossaryentry{performance_coefficient}{
    name={Performance Coefficient},
    description={The ratio of the power extracted by the turbine to the power
    available in the wind. See section \ref{sec:extracting_power_from_wind} for 
    details}
}

% Rayleigh Distribution
\newglossaryentry{rayleigh_distribution}{
    name={Rayleigh Distribution},
    description={A distribution that is often used to model wind speed. See section
    \ref{sec:windspeed_distributions} for details}
}

% Roughness Length
\newglossaryentry{roughness_length}{
    name={Roughness Length},
    description={A measure of the height above the surface at which the windspeed
    theoretically becomes zero. See section \ref{sec:windspeed_distributions} for
    details and table \ref{tab:roughness_lengths} for some example values}
}

% PV - photovoltaic
\newglossaryentry{PV}
{
    name=PV,
    description={Photovoltaic: Conversion of light into electricity using solar 
    cells.}
}

% SM - solar mass
\newglossaryentry{SM}
{
    name=SM,
    description={Solar Mass: The mass of the sun, used as a unit of mass in 
    astronomy.}
}

% Zenith angle
\newglossaryentry{zenith angle}
{
    name=Zenith Angle,
    description={The angle between the sun and the zenith (the point directly 
    overhead). See section \ref{sec:solar_definitions} for more details}
}

% Band gap
\newglossaryentry{band gap}
{
    name=Band Gap,
    description={The energy difference between the valence band (where electrons 
    are normally located) and the conduction band (a higher energy level where 
    electrons can move freely). See section \ref{sec:PV_fundamentals} for more 
    details.}
}

% CDR
\newglossaryentry{CDR}{
    name={CDR},
    description={Carbon Dioxide Removal, see section \ref{sec:CDR} for details}
}

% CCS
\newglossaryentry{CCS}{
    name={CCS},
    description={Carbon Capture and Storage, see section \ref{sec:CDR} for details}
}

% SRM
\newglossaryentry{SRM}{
    name={SRM},
    description={Solar Radiation Management, see section \ref{sec:SRM} for details}
}

% EATS
\newglossaryentry{EATS}{
    name={EATS},
    description={Effectiveness, Affordability, Timeliness, Safety, see section 
    \ref{sec:geoengineering} for details}
}

% NBS
\newglossaryentry{NBS}{
    name={NBS},
    description={Nature-Based Solutions}
}

% BECCS
\newglossaryentry{BECCS}{
    name={BECCS},
    description={Bio-Energy with carbon capture and storage, see section 
    \ref{sec:CDR} for details}
}