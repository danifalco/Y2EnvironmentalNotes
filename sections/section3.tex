\section{Radiative Forcing and Feedback}
\label{sec:forcing_feedback}
\subsection{Radiative Forcings and Feedbacks}
Let $I_N$ be the net irradiance at the \gls{TOA}, defined positive in the downward direction. It
follows that in \textbf{radiative equillibrium}:
$$
I_N = \frac{(1 - \alpha_p) \text{\gls{TSI}}}{4} - \text{\gls{OLR}} = 0
$$

\begin{itemize}
    \item A \textbf{radiative forcing} ($\Delta Q_{EXT}$) is an external perturbation to the climate system away from 
radiative equillibrium at the \gls{TOA}. A positive forcing leads to a positive global mean surface temperature
change ($\Delta T_s$), and vice versa.
    \item A \textbf{feedback} ($\Delta Q_{INT}$) is an internal process which responds to the forcing \textbf{and} has an
    effect on the global mean surface temperature.
\end{itemize}

\noindent It follows from the above definitions that:
\begin{align}
    \text{In the \textbf{absence} of feedback processes} \quad \Delta I_N &= \Delta Q_{EXT} \nonumber \\
    \text{In the \textbf{presence} of feedback processes} \quad \Delta I_N &= \Delta Q_{EXT} + \Delta Q_{INT} \nonumber
\end{align}

We can also define a further parameter, the \textbf{climate feedback parameter} $\lambda$. This is a measure 
of how much the planet's energy balance responds to changes in global mean surface temperature $\Delta T_s$. It is a
measure of how much additional energy is trapped or lost as a result to said change in surface temperature. We define
the total \textbf{climate feedback parameter} $\gamma$ as:
$$
\Delta Q_{INT} = \gamma \Delta T_s \label{eq:gamma}
$$
where $\gamma$ represents the radiative response of the climate feedback to a given $\Delta T_s$. Its units are 
$\text{W m}^{-2} \text{K}^{-1}$.

\subsection{Decomposing the Feedback Parameter}
\label{sec:decompose_feedback}
We can decompose the total feedback parameter $\gamma$ into the sum of individual feedback parameters:
$$
\gamma = \frac{dI_N}{dT_s} = \frac{\partial I_N}{\partial T_s} + \sum_x \frac{\partial I_N}{\partial x} 
\frac{\partial x}{\partial T_s}
$$
where $x$ represents the various parameters (e.g. water vapour). Here, $\frac{\partial I_N}{\partial T_s} = -\gamma_{BB}$
is the \textbf{blackbody feedback parameter}. The negative sign is simply to represent the fact that the surface will 
radiate more more energy as it warms, reducing the net irradiance at the \gls{TOA} ($I_N$).

\begin{tcolorbox}
    \textbf{Example:\\}
    For a simple 1 layer atmosphere transparent to solar radiation, it follows that $\text{\gls{OLR}} = \epsilon ' \sigma T_s^4$
    where $\epsilon '$ is the effective \hyperlink{glo:emissivity}{emissivity} which reduces with atmospheric 
    \hyperlink{glo:absorptivity}{absorptivity} and hence \hyperlink{glo:emissivity}{emissivity} increases. We can write:
    $$
    \frac{\partial I_N}{\partial T_S} = -\frac{\partial\ \text{\gls{OLR}}}{\partial T_s} = -\gamma_{BB} \quad \implies \quad
    \boxed{\gamma_{BB} = 4 \epsilon ' \sigma T_s^3}
    $$
\end{tcolorbox}

As per the example above, it becomes apparent that for a parameter ($x$) to exert a strong feedback, it must be dependent
on both $T_s$ \textbf{and have an influence on} the net irradiance at the \gls{TOA} ($I_N$). 

\subsection{Water Vapour Feedback and Clausius-Clapeyron Scaling}
\label{sec:clausius-clapeyron}

We mention in section \ref{sec:greenhouse_definition} that water vapour is the main absorber or \gls{LW} radiation, and 
therefore the most important greenhouse gas. Water vapour concentrations show a strong dependence on temperature, i.e.
the warmer the atmosphere, the more water vapour it can hold. This is a result of the Clausius-Clapeyron relation.\\

As per Paterson's thermodynamics course, we know that the pressure of water vapour which is saturated with respect to 
liquid water at a given temperature $T$ is given by:
$$
\frac{dP_s}{dT} = \frac{s_v - s_l}{v_v - v_l} \quad \implies \quad \boxed{\frac{dP_s}{dT} = \frac{l_v}{T v_v}}
$$
at the vapour-liquid phase boudary where $s_v$ and $s_l$ are the specific entropies of vapour and liquid water respectively,
$v_v$ and $v_l$ are the specific volumes of vapour and liquid water respectively, $l_v$ is the latent heat of
vaporisation and $P_s$ is the saturation vapour pressure.

If we assume water vapour behaves as an ideal gas, it is apparent that $P_s v_v = Nk_B T/Nm_v = k_B T / m_v$ where $m_v$ is
the molecular mass of water (remember we are using specific volume, not volume). We can define the gas constant for water
vapour $R_v = R_v = k_B / m_v$ and hence:
$$
\frac{dP_s}{dT} = \frac{P_s l_v}{R_v T^2} \quad \implies \quad \frac{dP_s}{P} = \frac{l_v}{R_v T^2}dT
$$
for context, we find that for sensible values for the constants:
$$
\frac{\left(\frac{dP_s}{P_s}\right)}{dT_s} \approx 0.07\ \text{K}^{-1}
$$
we call this the Clausius-Clapeyron scaling. Note that this does not apply to the actual vapour pressure (related to
the absolute amount of water vapour in the atmosphere). Vapour pressure and saturation vapour pressure are related by:
$$
\text{Relative Humidity, RH} = \frac{P_v}{P_s} \times 100\% \quad \implies \quad +7\%\ \text{RH per degree warming}
$$
where $P_v$ is the vapour pressure. 

Climate models suggest that relative humidity is conserved under anthropogenic climate change meaning that the absolute 
amount of water vapour in the atmosphere increases as atmospheric temperature increases.
