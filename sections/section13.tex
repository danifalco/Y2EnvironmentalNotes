\section{Geoengineering Our Climate}
\label{sec:geoengineering}

For this course we will take the following definition of geoengineering:
\begin{tcolorbox}
    \centering
    \textit{``The deliberate, large-scale intervention in the 
    Earth's natural systems to counteract climate change.''}
\end{tcolorbox}

Two main types of geoengineering are considered:
\begin{itemize}
    \item \textbf{Carbon Dioxide Removal (CDR)}: The removal of \ce{CO2} from 
    the atmosphere in an attempt to minimise the atmosphere's overall greenhouse
    effect and hence reduce global mean surface temperature. This increases the
    OLR side of the energy-balance equation. Examples include afforestation,
    ocean fertilisation and direct air capture.
    \item \textbf{Solar Radiation Management (SRM)}: The reflection of incoming
    solar radiation away from the Earth in an attempt to reduce the amount of
    energy absorbed by the Earth's surface. This reduces the absorbed solar
    radiation side of the energy-balance equation. Examples include stratospheric
    aerosol injection, low cloud albedo enhancement and space-based reflectors.
\end{itemize}

Before we discuss the various methods of geoengineering, we must introduce a 
concept that we will have to bear in mind throughout the rest of this section - 
\textbf{EATS}. This stands for the following:

\begin{itemize}
    \item \textbf{E}ffectiveness: Is the proposed method going to have a significant,
    measurable impact on the climate system?
    \item \textbf{A}ffordability: Is the proposed method likely to be affordable?
    \item \textbf{T}imeliness: Is the proposed method likely to be available in
    time to prevent dangerous climate change? How mature is this technology?
    \item \textbf{S}afety: Is the proposed method likely to have any negative
    side-effects?
\end{itemize}
And lastly (not in this mnemonic) we should also consider legality, and feasibility,
international cooperation, public perception etc.

\subsection{Carbon Dioxide Removal}
\label{sec:CDR}

In this section we will discuss various examples of \gls{CDR}. \\

\noindent \textbf{Bio-Energy with carbon capture and storage (BECCS)}: This is a combination
of two technologies - bio-energy and \gls{CCS}. The idea is to grow biomass (trees), burn
it to generate energy and then capture the \ce{CO2} emitted during combustion through 
\gls{CCS} techniques. The captured \ce{CO2} is then transported and stored underground
or under the sea. This process therefore acts as a net \textbf{carbon sink} - it removes
\ce{CO2} from the atmosphere and stores it underground. This is sometimes referred 
to as ``negative emissions''. 

However, it is worth noting that implementing this is not without its challenges.
Mainly, this requires a vast amount of land to grow the biomass on, which 
negatively impacts food security and biodiversity. Furthermore, the time-scale
is not particularly favourable and there are significant monetary costs involved.\\

\noindent \textbf{Enhanced Rock Weathering}: This is a climate change mitigation 
strategy that accelerates the natural process of weathering to remove carbon 
dioxide from the atmosphere. It involves pulverizing rocks rich in silicate 
minerals, like basalt or olivine, and spreading them over large areas. The 
increased surface area of the powdered rock allows for faster chemical reactions 
with \ce{CO2}, locking it away in stable forms such as bicarbonate ions. This not 
only reduces atmospheric \ce{CO2} levels but also replenishes soil nutrients for 
crops and promotes plant growth, which further absorbs \ce{CO2} (feedback). Furthermore,
when eventually these are washed away by rivers into the ocean these are the ``right
kind'' of carbonates that help reduce ocean acidification.
Despite its potential, the method poses challenges, including the energy-intensive 
process of mining and grinding rocks and the potential reduction of albedo due 
to the dark rocks.\\

\subsection{Solar Radiation Management}
\label{sec:SRM}

\noindent \textbf{Marine Cloud Brightening}: This is a proposed form of \gls{SRM} 
that levereges the Twomey effect (see Section \ref{sec:aerosol-radiative-forcing}),
i.e. the indirect effect. The idea is to spray aerosols into the atmosphere which,
in areas of marine stratocumulus (this occurs especially to the west of continents)
will increase the number of cloud droplets and hence increase the cloud albedo 
significantly. This is suggested to be done by having a fleat of ships that pump
sea salt into the atmosphere. This will increase the amount of solar radiation 
reflected back
into space, reduce the amount of solar radiation absorbed by the Earth's surface
and therefore reduce global mean surface temperature. \\

However, this method is not without its challenges. It is not clear whether
this system is really scalable. Secondly, there might be hidden carbon costs 
(having a fleat of ships pumping sea salt into the atmosphere is not exactly
carbon neutral). 
